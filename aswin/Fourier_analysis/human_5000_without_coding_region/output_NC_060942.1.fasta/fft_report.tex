\documentclass{article}
\usepackage{graphicx}
\usepackage{longtable}
\usepackage{geometry}
\usepackage{float}
\usepackage{hyperref}
\geometry{margin=1in}
\title{Fourier Spectral Analysis of Coding Sequences}
\author{Automated Report}
\date{\today}
\begin{document}
\maketitle
\tableofcontents
\newpage
\section{Group 1: Single Peak at 0.33}
\section{Group 2: Single Peak Elsewhere}
\section{Group 3: Multiple Peaks}
\subsection{SCFR\_NC\_060942.1\_74425633\_74430805}\label{sec:3_SCFR_NC_060942_1_74425633_74430805}
\subsubsection{Protein: frame-2}
\textbf{Gene ID}: SCFR\\
\textbf{Nucleotide Accession}: NC\_060942.1\\
\textbf{Sequence Length}: 5172 bp\\
\textbf{Protein Description}: \\ \begin{minipage}[t]{\linewidth} Stop-codon-free region (74425633-74430805) \end{minipage} \\ 
\begin{figure}[H]\centering
\includegraphics[width=0.9\linewidth]{SCFR_NC_060942_1_74425633_74430805_frame_2_fft.png}
\caption{FFT Spectrum for frame-2}
\end{figure}
\begin{longtable}{|c|c|}
\hline
Frequency & Motifs\\\hline
0.481 & GG (2074), GA (942), AG (766), TG (445), GT (256)\\\hline
0.296 & GGG (1115), GGA (588), AGG (578), GAG (574), TGG (372)\\\hline
0.111 & GGGTGGGA (189), GGTGGGAG (188), AGGGTGGG (185), GGGGGATG (179), GGGGATGG (176)\\\hline
\end{longtable}
\newpage
\end{document}
