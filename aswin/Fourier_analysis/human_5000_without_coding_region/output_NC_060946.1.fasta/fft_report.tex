\documentclass{article}
\usepackage{graphicx}
\usepackage{longtable}
\usepackage{geometry}
\usepackage{float}
\usepackage{hyperref}
\geometry{margin=1in}
\title{Fourier Spectral Analysis of Coding Sequences}
\author{Automated Report}
\date{\today}
\begin{document}
\maketitle
\tableofcontents
\newpage
\section{Group 1: Single Peak at 0.33}
\section{Group 2: Single Peak Elsewhere}
\section{Group 3: Multiple Peaks}
\subsection{SCFR\_NC\_060946.1\_20758991\_20765624}\label{sec:3_SCFR_NC_060946_1_20758991_20765624}
\subsubsection{Protein: frame3}
\textbf{Gene ID}: SCFR\\
\textbf{Nucleotide Accession}: NC\_060946.1\\
\textbf{Sequence Length}: 6633 bp\\
\textbf{Protein Description}: \\ \begin{minipage}[t]{\linewidth} Stop-codon-free region (20758991-20765624) \end{minipage} \\ 
\begin{figure}[H]\centering
\includegraphics[width=0.9\linewidth]{SCFR_NC_060946_1_20758991_20765624_frame3_fft.png}
\caption{FFT Spectrum for frame3}
\end{figure}
\begin{longtable}{|c|c|}
\hline
Frequency & Motifs\\\hline
0.333 & CG (1045), GC (835), CC (657), GG (620), GA (478)\\\hline
0.333 & CGC (543), ACG (390), AGG (387), GCC (359), TCG (284)\\\hline
0.074 & AACGAGGACGCCG (76), TCGCTAACGAGGA (58), AGGACGCCGCCCA (56), GTCGCTAACGAGG (54), GAGGACGCCGCCC (53)\\\hline
\end{longtable}
\newpage
\subsection{SCFR\_NC\_060946.1\_20759017\_20765623}\label{sec:3_SCFR_NC_060946_1_20759017_20765623}
\subsubsection{Protein: frame-2}
\textbf{Gene ID}: SCFR\\
\textbf{Nucleotide Accession}: NC\_060946.1\\
\textbf{Sequence Length}: 6606 bp\\
\textbf{Protein Description}: \\ \begin{minipage}[t]{\linewidth} Stop-codon-free region (20759017-20765623) \end{minipage} \\ 
\begin{figure}[H]\centering
\includegraphics[width=0.9\linewidth]{SCFR_NC_060946_1_20759017_20765623_frame_2_fft.png}
\caption{FFT Spectrum for frame-2}
\end{figure}
\begin{longtable}{|c|c|}
\hline
Frequency & Motifs\\\hline
0.333 & CG (1042), GC (831), CC (655), GG (619), GA (477)\\\hline
0.333 & CGC (541), ACG (390), AGG (386), GCC (358), TCG (282)\\\hline
0.074 & AACGAGGACGCCG (76), TCGCTAACGAGGA (58), AGGACGCCGCCCA (56), GTCGCTAACGAGG (54), GAGGACGCCGCCC (53)\\\hline
\end{longtable}
\newpage
\end{document}
